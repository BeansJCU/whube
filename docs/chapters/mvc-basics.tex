\chapter {MVC Basics of Whube}
This chapter will cover the basics of the whube MVC
basics. This will outline how Whube works under the hood.
So, let's start off slow, and take a look at a simple incoming
request.
\section {Simple Request}
So. Let's deal with the incoming request \texttt{http://whube.com/t/foo}.
This request is sent to the server \texttt{whube.com} over the \texttt{http}
protocol. CS 101, I know. Let's now look at the part handled by the Whube
application. The request to the Whube webapp is \texttt{/t/foo}. This is handled
by the .htaccess file in the project directory. The rule reads:
\begin{verbatim}
    RewriteRule ^t/(.*+)      controller.php?p=$1   [NC]
\end{verbatim}
This means that any requests that start with \texttt{/t/} will be passed to controller.php
with everything after the last slash as the GET['p'] argumnt. \\
\\
The \texttt{/t/foo} request is just an alias for \texttt{controller.php?p=foo}
\\
the \texttt{controller.php?p=foo} script then goes on to do a few things. The \texttt{controller.php}
script includes the \texttt{conf/site.php} and \texttt{libs/php/globals.php} files for use by the 
content script. \texttt{controller.php} then searches the \texttt{content/} directory for a content
script that matches the request \texttt{foo}. \texttt{controller.php} resolves this to \texttt{content/foo.php}.
If this script does not exist, it will default to \texttt{/t/default}.
\\
\\
If \texttt{content/foo.php} exists, it will be included by \texttt{controller.php}.
This script defines two varables. \texttt{\$CONTENT} and \texttt{\$TITLE}.
\texttt{\$CONTENT} is the "meat" of the page, and \texttt{\$TITLE} is the HTML
page title ( \texttt{<title>} ).
After the controller includes the content script, in invokes \texttt{view/view.php}.
This will take all the varables that the \texttt{content/foo.php} script sets up, and
echos them to the screen in a pre-created HTML template.
\\
\\
\section{Hello, World!}
Let's take a look at the most simple content script. It's also pretty cool.
\begin{verbatim}
<?php
    $TITLE   = "Hello, World!"; // title of the page
    $CONTENT = "Hello, World!"; // content on the page
?>
\end{verbatim}
Well, if you don't understand this, you might want to go back and review PHP. This
is a very straight forward example, and it only goes downhill from here!
